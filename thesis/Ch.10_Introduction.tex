\chapter{Introduction\label{intro}}
Node.js is a asynchronous event-driven JavaScript runtime environment that is designed to build scalable network applications (\cite{node.jsAbout}).
In its core process it uses event loop at runtime to handle any callbacks.
Node.js is able to perform blocking I/O event within a single runtime thread by offloading operations to system kernel when possible (\cite{node.jsEventLoop}).
These operations are called non-blocking operations or asynchronous operations (\cite{node.jsOverviewBlockVsNonBlock}).
Events that block the event loop from continuing its process are called blocking operations or synchronous operations.
Node.js itself is able to perform one event at any given time and offloading processes to continue working on other events.

In this paper we study a case where a node.js component runs inside a monolith application and is refactored to be a part of microservice architecture.
To the case it is critical that the studied component is as fast as it can be.
We are going to study if there is a difference in the components response time in these environments and if the single threaded event loop of node.js has any part to it.

There are many studies related to the performance of node.js application.
Many studies compared node.js server performance to other programming languages.
\cite{Challapalli}, \cite{Lion} and many others compared performance of node.js server to other servers build with different programming languages including python, .net, java, c++, php etc.
These studies mainly focused on the throughput and response time of a servers application interface with different user scenarios.

A deeper review of node.js and the event loop is given in the background chapter.
In the case chapter and overview of the case architecture is given including a review of the studied component.
Methods for collecting components response time and literature collection methods are described in the methods chapters and their results in the following results chapter.
Discussion to found results and their meaning is found in the related chapter.

