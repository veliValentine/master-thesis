\chapter{Methods\label{methods}}

\section{RQ}

Research questions

RQ1: What is the response time in server side monolith JavaScript application?

RQ2: What is the response time in server side JavaScript service?

\section{Literature}
To find out about existing literature related to the case a two phase hybrid search was performed.
First phase contained a systematic literature review with predefined keywords.
The seconds phase used snowballing method with one iteration on the literature found on first phase.
One iteration consisted one back step and one forward step for each final results from phase one.

\subsection{Literature review}
The first search phase is a literature review to find out relevant literature to the case.

The review was performed in the following steps.
\begin{enumerate}
    \item Define the review protocol.
    \item Search literature based on the protocol.
    \item Review and analyze each results.
    \item Combine all the relevant results.
\end{enumerate}

\subsubsection{Review protocol}
The search string.
Golden quadstring???

Results analysis.
Point system for inclusion/exclusion.


\subsubsection{The search}
We used search string \textbf{SEARCH STRING} in the following search engines.
\begin{itemize}
    \item List of search engines
\end{itemize}
We found \textbf{X} articles from the search

\subsubsection{Review and analysis}
Each article was review based on the review criteria. Each criteria was pointed based on the point system.

For reviewing the relevance of the articles each articles title and abstract was reviewed.

\begin{flushleft}
\begin{tabular}{|c c|} 
 \hline
 Criteria & Description \\ [0.5ex] 
 \hline
  Cr.1 & JavaScript performance review  \\ 
  \hline
  Cr.2 & Node.js performance review  \\ 
  \hline
  Cr.3 & Architectural performance review  \\ 
  \hline
  Cr.4 & Response time performance review  \\ 
  \hline
  Cr.5 & Server side performance review  \\ 
  \hline
\end{tabular}
\end{flushleft}


\begin{flushleft}
\begin{tabular}{|c c|} 
 \hline
 Points & Description \\ [0.5ex] 
 \hline
  0 & No relevance to the criteria  \\ 
 \hline
  0.5 & Some relevance to the criteria \\ 
 \hline
 1 & Relevant to the criteria \\ 
 \hline
\end{tabular}
\end{flushleft}

\textbf{A table of results and total points}

\subsubsection{Final result}
For the final results only articles with the relevance point at or above \textbf{relevance point threshold} was included.

\subsection{Snowballing - Phase two}

\section{Data}
% What data was collected?
To measure the response time of the case module a series of data points were collected from the module.
The data points were collected from two environments.
First environment was integrated as a part of the monolith application.
Second environment contained the module and minimum amount of other functionalities to make it work as service.

The measured data contains run-time information about the module.
A single data point contains five values. 
\begin{enumerate}
    \item Cycle id
    \item Timestamp id
    \item System time (in nanoseconds)
    \item System free memory
    \item Used process CPU
\end{enumerate}

The cycle id, system time in nanoseconds, system free memory, and used process CPU.
The cycle id is unique for each full cycle.
The memory and CPU usages were collected to ensure that the system had enough resources to operate.
Data points were collected at the beginning of the application cycle, at the end of the cycle and before and after each third party or internal function call.

For each environment the data was logged to \textit{stdout} stream where it was collected for later analysis.
Since sometimes the cycle can be interrupted by an error only data points from full cycles were included. A full cycle contains start and end data point of a cycle.

\textbf{Data about how many total data points per environment and how many full cycles}




% What data was included and what excluded? And reasons

% What data sources are used and why?


% How data from the modules was collected

% How data was analysed

% How the data can answer the research questions

% How case size (data points) was defined


